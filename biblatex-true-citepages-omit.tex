\documentclass{ltxdockit}[2011/03/25]
\usepackage{btxdockit}
\usepackage[main=english,french,latin]{babel}
\usepackage[bibstyle=verbose,citestyle=verbose,citepages=omit]{biblatex}
\addbibresource{biblatex-true-citepages-omit.bib}
\usepackage{biblatex-true-citepages-omit}
\savebibmacro{real-citepages-omit}
\renewbibmacro{real-citepages-omit}{}
\usepackage{fontspec}
\usepackage[mono=false]{libertine}
\usepackage{microtype}
\usepackage[strict]{csquotes}
\setmonofont[Scale=MatchLowercase]{DejaVu Sans Mono}
\usepackage{shortvrb}
\usepackage{minted}
\usepackage{pifont}

% Usefull commands
\newcommand{\biblatex}{biblatex\xspace}
\pretocmd{\bibfield}{\sloppy}{}{}
\pretocmd{\bibtype}{\sloppy}{}{}
\newcommand{\bibkey}[1]{\texttt #1}
% Meta-datas
\titlepage{%
	title={Correcting a limitation of citepages=omit option of \biblatex styles},
	email={maieul <at> maieul <dot> net},
	author={Maïeul Rouquette},
	subtitle={},
	revision={1.0.0},
	date={25/03/2014},
	url={https://github.com/maieul/biblatex-true-citepages-omit}}
% Hyphen
\hyphenation{Ma-de-lei-ne}

\begin{document}

\printtitlepage
\tableofcontents

\section{Documentation}

\subsection{Goal}
The citation styles of the family \emph{verbose} of the \biblatex package offer an \emph{citepages=omit} option.
\begin{minted}{tex}
\usepackage[bibstyle=verbose,citestyle=verbose,citepages=omit]{biblatex}
\end{minted}

This option is very useful: it omits the \bibfield{pages} field when a citation is made with a pages' number. For example, the following entry

\inputminted{tex}{biblatex-true-citepages-omit.bib}

called with:

\begin{minted}{tex}
\cite[112]{Kaestli1993}
\end{minted} 

will be printed as:

\begin{quotation}
\cite[112]{Kaestli1993}
\end{quotation}

and not as:
\citereset

\begin{quotation}
\cite{Kaestli1993}, p.~112
\end{quotation}

This option works perfectly, except if you add a text after page's number, as you can see in the following example.

\begin{minted}{tex}
\cite[\pno~112, on this point etc.]{Kaestli1993}
\end{minted}

\citereset
\begin{quotation}
\cite[\pno~112, on this point etc.]{Kaestli1993}
\end{quotation}

It seems this problem won't be solved in the \biblatex core\footnote{See: \url{https://github.com/plk/biblatex/issues/151\#issuecomment-22879677}.}.

So, if you simply load the \emph{biblatex-true-citepages-omit} footcite package the problem will be solved.

\begin{minted}{tex}
\usepackage[bibstyle=verbose,citestyle=verbose,citepages=omit]{biblatex}
\usepackage{biblatex-true-citepages-omit}
...
\cite[\pno~112, on this point etc.]{Kaestli1993}
\end{minted}


\citereset
\restorebibmacro{real-citepages-omit}{}
\begin{quotation}
\cite[\pno~112, on this point etc.]{Kaestli1993}
\end{quotation}
\subsection{Declare commands to be used in \prm{postnote} argument}

If you use some commands in the \prm{postnote} arguments, it could kill \LaTeX. To prevent that, use the \cmd{realcitepagesomit}, which the name of the command as argument. For example, the \emph{polyglossia} switching commands are problematic. So you can do:
\begin{minted}{tex}
\realcitepagesomitcmd{textlatin}
\realcitepagesomitcmd{textgreek}
\end{minted}
\subsection{Limitations}

\begin{itemize}
	\item This package works only if you start the  \prm{postnote}  argument by the \cmd{pno} or \cmd{ppno} commands.
	\item This package must be loaded after the \emph{biblatex-source-division} package.
\end{itemize}
\section{Credits}

This package was created for Maïeul Rouquette's phd dissertation\footnote{\url{http://apocryphes.hypothese.org}.} in 2014. It is licensed on the \emph{\LaTeX\ Project Public License}\footnote{\url{http://latex-project.org/lppl/lppl-1-3c.html}.}. 


All issues can be submitted, in French or English, in the GitHub issues page\footnote{\url{https://github.com/maieul/biblatex-source-division/issues}.}.



\section{Revision History}
\begin{changelog}




\begin{release}{1.0.0}{2014-13-25}
\item First public release.
\end{release}

\end{changelog}
\end{document}